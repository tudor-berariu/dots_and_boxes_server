\documentclass[12pt]{article}

\usepackage[utf8x]{inputenc}
\usepackage{hyperref}
\usepackage{minted}
\usepackage{graphicx}

\title{Inteligență Aritificială \\ Tema 1: \emph{Biscuitele}}
\author{Tudor Berariu \\ \emph{tudor.berariu@gmail.com}}

\begin{document}
\maketitle

\section{Descrierea jocului}
\label{sec:desc}

Jocul \emph{Dots and Boxes} se desfășoară pe o matrice de dimensiune
$height \times width$. Importante sunt cele $(height+1) \times
(width+1)$ puncte de intersecție. În timpul jocului, pe rând, fiecare
dintre cei doi jucători unește două puncte vecine cu o linie
orizontală sau verticală. Scopul este acela de a închide cât mai multe
celule prin unirea puncelor. Fiecare jucător primește un punct pentru
fiecare celulă pe care îl închide (toți cei patru pereți care
delimitează celula au fost adăugați). La finalul jocului câștigă acel
jucător care a strâns mai multe puncte.

\begin{figure}[h]
  \centering
  \includegraphics[width=.3\textwidth]{dots.png}
  \caption{Image from Wikipedia}
\end{figure}

Jucătorii \emph{mută} alternativ, dar un jucător care reușește în urma
unei mutări să închidă o celulă mai primește dreptul la încă o mutare.



\section{Cerințe și notare}
\label{sec:tasks}

Implementați algoritmul MiniMax pentru jocul \emph{Dots and
  Boxes}. Algoritmul trebuie completat de tehnica $\alpha-\beta$
pruning. Găsiți o euristică cât mai bună astfel încât să bateți cei
doi jucători implementați deja.

Există o limită de timp de o secundă pe mutare.

Se vor acorda 5 puncte pentru implementarea algoritmului MiniMax cu
$\alpha-\beta$ pruning.

Se vor acorda 5 puncte pentru găsirea unei euristici destul de bune
încât să bată jucătorii pe care îi găsiți în arhiva temei.

Se vor acorda până la 2 puncte bonus pentru cele mai bune 50\% dintre
teme. Toate soluțiile trimise vor fi confruntate, se va face un
clasament general, iar prima jumătate din acel clasament va primi de
la 2 la 0.5 puncte (descrescător în ordinea punctajului). Clasamentul
va fi făcut întâi după numărul de jocuri câștigate și apoi după
\emph{golaverajul} general.

\section{Arhiva temei}
\label{sec:arhiva}

Arhiva temei conține un script \texttt{game\_server.py} care confruntă
toți jucătorii din directorul \texttt{players}. Fiecare jucător este
implementat într-un fișier separat în care se găsește o clasă cu
același nume.

Pentru a vă testa soluția, implementați o soluție, puneți fișierul în
directorul \texttt{players} și rulați \texttt{make clean} și
\texttt{make}. Dacă aveți \texttt{pdflatex} și desktop Gnome se va
deschide un pdf cu clasamentul jucătorilor.

\section{Trimiterea temei}
\label{sec:archive}

Arhiva temei va conține un fișier \texttt{PDF} cu descrierea soluției
folosite în temă și un \emph{singur} fișier \texttt{Python} cu
implementarea soluției. Fișierele vor avea un nume construit astfel:
\texttt{NumePrenumeAAAALLZZ.ext} din numele complet și data
nașterii. Renunțați, desigur la diacritice și la linii. \texttt{ext}
va fi \texttt{pdf} sau \texttt{py}.

În fișierul \texttt{Python} se va găsi o clasă cu același nume în care
vor fi cel puțin următoarele două metode: \texttt{\_\_init\_\_} și \texttt{move}.

\begin{minted}{python}
  class RandomPlayer:
    def __init__(self):
      self.name = "Random Player"

    def move(self, board, score):
      return (0,0)
\end{minted}

Metoda \texttt{\_\_init\_\_} va inițializa un câmp \texttt{name} cu un șir
de caractere ce conține numele complet.

Metoda \texttt{move} primește două argumente: \texttt{board} și \texttt{score}:
\begin{itemize}
\item \texttt{board} este o listă cu $height * 2 + 1$ liste. Cele de
  pe pozițiile pare ($0, 2, \ldots, height * 2$) corespund liniilor
  orizontale și au lugime $width$. Listele de pe pozițiile impare
  corespund liniilor verticale și au lungime $width + 1$.
\item \texttt{score} reprezintă un tuplu cu scorul curent: punctajul
  propriu, punctajul adversarului.
\end{itemize}

\paragraph{Exemplu de matrice}

Pentru matricea următoare:
\begin{verbatim}
*-* * *
| |   |
* * *-*
  | |  
* * * *
|      
*-*-*-*
\end{verbatim}
parametrul \texttt{board} va fi:
\begin{verbatim}
[[1, 0, 0],
 [1, 1, 0, 1],
 [0, 0, 1],
 [0, 1, 1, 0],
 [0, 0, 0],
 [1, 0, 0, 0],
 [1, 1, 1]]
\end{verbatim}

În arhiva pe care o încărcați pe \url{curs.cs} puneți direct cele două
fișiere (nu un director).

\end{document}
